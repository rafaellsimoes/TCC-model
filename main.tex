%% Este trabalho é uma adequação das normas de TCC
%% da Universidade Federal de Mato Grosso (Faculdade de Engenharia) 
%% de acordo com a Norma ABNT.

%% ========================================================================
%% Opções: \documentclass[tipo/curso]{faeng}
%% ------------------------------------------------------------------------
%% Tipo:
%% 	tcc:         Formata documento para TCC
%%	qualitcc:    Formata documento para qualificação de TCC
%% ------------------------------------------------------------------------
%% Curso:
%%   eq:     Engenharia Química
%%   em:     Engenharia de Minas
%%   ec:     Engenharia de Computação
%%   et:     Engenharia de Transportes
%%   eca:    Engenharia de Controle e Automação
%% ------------------------------------------------------------------------
\documentclass[tcc/eca]{faeng}
%% ========================================================================

%% ========================================================================
%% PACOTES
%% ------------------------------------------------------------------------
%% Pacotes fundamentais 
%% ------------------------------------------------------------------------
\usepackage{cmap}			% Mapear caracteres especiais no PDF
\usepackage{lmodern}		% Usa a fonte Latin Modern			
\usepackage{makeidx}        % Cria o indice
\usepackage{hyperref}  		% Controla a formação do índice
\usepackage{lastpage}		% Usado pela Ficha catalográfica
\usepackage{indentfirst}	% Indenta o primeiro parágrafo de cada seção.
\usepackage{nomencl} 		% Lista de simbolos
\usepackage{graphicx}		% Inclusão de gráficos
\usepackage[brazil]{babel}  % Codificação e uso de caracteres especiais.
\usepackage[utf8]{inputenc}
%% ------------------------------------------------------------------------
%% Pacotes adicionais, usados apenas no âmbito do Modelo faeng
%% ------------------------------------------------------------------------
\usepackage{lipsum}				       % para geração de dummy text
\usepackage[printonlyused]{acronym}
\usepackage{xcolor}
\usepackage{booktabs}
\usepackage{multirow}
%% ========================================================================

%% ========================================================================
%% Informações de dados para CAPA e FOLHA DE ROSTO
%% ------------------------------------------------------------------------
%% Título:
%%	1. Título em português
%%	2. Título em inglês
\titulo{Título}{Title}
%% ------------------------------------------------------------------------
%% Autor:
%%	1. Nome completo do autor
%%	2. Formato de nome para bibliografia
\autor{Nome Completo}{Sobrenome, Nome}
%% ------------------------------------------------------------------------
%% Cidade
\local{Várzea Grande}
%% ------------------------------------------------------------------------
%% Ano de defesa
\data{2019}
%% ------------------------------------------------------------------------
%% Nome do orientador
\orientador{Nome completo do Orientador}
%% Nome do coorientador
%\coorientador{Nome completo do coorientador}
%% ========================================================================

%% ========================================================================
%% compila o indice
%% ------------------------------------------------------------------------
\makeindex
%% ========================================================================

%% ========================================================================
%% Compila a lista de abreviaturas e siglas
%% ------------------------------------------------------------------------
\makenomenclature
%% ========================================================================

%% ========================================================================
%% Inserir ficha catalográfica
%%
%% Caso o comando \inserirfichacatalografica seja definido, a ficha catalográfica
%% será inserida atrás da folha de rosto. Caso contrário a página será deixada em branco.
%%
%% CUIDADO: Esta opção deve ser preenchida antes do comando \maketitle
%% ------------------------------------------------------------------------
%\inserirfichacatalografica{fichaCatalografica.pdf}
%% ========================================================================

%% ========================================================================
%% Inserir folha de aprovação
%%
%% Caso o comando \inserirfolhaaprovacao seja definido, a a folha de aprovação
%% será inserida.
%% CUIDADO: Esta opção deve ser preenchida antes do comando \maketitle
%% ------------------------------------------------------------------------
%\inserirfolhaaprovacao{folhaAprovacao.pdf}
%% ========================================================================

%% ========================================================================
%% INÍCIO DO DOCUMENTO
%% ------------------------------------------------------------------------
\begin{document}
%% ------------------------------------------------------------------------
%% ELEMENTOS PRÉ-TEXTUAIS
%% ------------------------------------------------------------------------
\pretextual
%% ------------------------------------------------------------------------
%% Insere Capa, Folha de rosto, Ficha catalográfica (se inserida)
%% e folha de aprovação (se inserida).
%% ------------------------------------------------------------------------
\maketitle
%% ------------------------------------------------------------------------
%% Dedicatória
%% ------------------------------------------------------------------------
\imprimirdedicatoria{Dedicatória.}
%% ------------------------------------------------------------------------
%% Agradecimentos
%% ------------------------------------------------------------------------
\imprimiragradecimentos{
Agradecimentos.
}
%% ------------------------------------------------------------------------
%% Epígrafe
%% ------------------------------------------------------------------------
\imprimirepigrafe{
		``Frase.''\\
		(Autor)
}
%% ========================================================================

%% ========================================================================
%% RESUMO e ABSTRACT
%% ------------------------------------------------------------------------
%% Resumo em português
%% ------------------------------------------------------------------------ 
\begin{resumo}{Palavra-chave. Palavra-chave. Palavra-chave. Palavra-chave. Palavra-chave. Palavra-chave}

 	 Resumo.

\end{resumo}
%% ------------------------------------------------------------------------
%% Resumo em inglês
%% ------------------------------------------------------------------------
\begin{abstract}{Keyword. Keyword. Keyword. Keyword. Keyword. Keyword}

Abstract.
	
\end{abstract}
%% ========================================================================

%% ========================================================================
%% inserir lista de ilustrações
%% ------------------------------------------------------------------------
\listailustracoes
%% ========================================================================

%% ========================================================================
%% inserir lista de tabelas
%% ------------------------------------------------------------------------
\listatabelas
%% ========================================================================

%% ========================================================================
%% inserir lista de abreviaturas e siglas
%% ------------------------------------------------------------------------
\listasiglas{abreviaturas}
%% ========================================================================

%% ========================================================================
%% inserir o sumario
%% ------------------------------------------------------------------------
\sumario
%% ========================================================================

%% ========================================================================
%% ELEMENTOS TEXTUAIS
%% ------------------------------------------------------------------------
\mainmatter
%% ========================================================================

%% ========================================================================
%% Capitulos 
%% ------------------------------------------------------------------------
%% Capítulo externo
\chapter[Introdução]{Introdução}

Este documento e seu código-fonte são exemplos de referência de uso da classe \textsf{faeng.cls} (baseada na classe \textsf{eesc.cls}) e do pacote \textsf{abntex2cite}. O documento exemplifica a elaboração de trabalho acadêmico produzido conforme a \ac{ABNT} \ac{NBR} 14724:2011 \emph{Informação e documentação - Trabalhos acadêmicos - Apresentação}.

Este modelo é uma implementação das normas para produção de textos estabelecida pela Faculdade de Engenharia da Universidade Federal de Mato Grosso, Campus Universitário de Várzea Grande.
%% ------------------------------------------------------------------------
%% Capítulo
\chapter[Capítulo]{Capítulo}
Esse também é um capítulo.
%% ------------------------------------------------------------------------
%% Capítulo
%% abtex2-modelo-include-comandos.tex, v-1.4 laurocesar
%% Copyright 2012-2013 by abnTeX2 group at http://abntex2.googlecode.com/ 
%%
%% This work may be distributed and/or modified under the
%% conditions of the LaTeX Project Public License, either version 1.3
%% of this license or (at your option) any later version.
%% The latest version of this license is in
%% abtex2-modelo-include-comandos.tex, v-1.4 laurocesar
%% Copyright 2012-2013 by abnTeX2 group at http://abntex2.googlecode.com/ 
%%
%% This work may be distributed and/or modified under the
%% conditions of the LaTeX Project Public License, either version 1.3
%% of this license or (at your option) any later version.
%% The latest version of this license is in
%%   http://www.latex-project.org/lppl.txt
%% and version 1.3 or later is part of all distributions of LaTeX
%% version 2005/12/01 or later.
%%
%% This work has the LPPL maintenance status `maintained'.
%% 
%% The Current Maintainer of this work is the abnTeX2 team, led
%% by Lauro César Araujo. Further information are available on 
%% http://abntex2.googlecode.com/
%%
%% This work consists of the files abntex2-modelo-include-comandos.tex
%%

% ---
% Este capítulo, utilizado por diferentes exemplos do abnTeX2, ilustra o uso de
% comandos do abnTeX2 e de LaTeX.
% ---
 
\chapter{Resultados de comandos}\label{cap_exemplos}

\chapterprecishere{Isto é uma sinopse de capítulo. A ABNT não traz nenhuma
normatização a respeito desse tipo de resumo, que é mais comum em romances 
e livros técnicos.}\index{sinopse de capítulo}

% ---
\section{Citações}
% ---

\index{citações!diretas}Utilize o ambiente \texttt{citacao} para incluir
citações diretas com mais de três linhas:

\begin{citacao}
As citações diretas, no texto, com mais de três linhas, devem ser
destacadas com recuo de 4 cm da margem esquerda, com letra menor que a do texto
utilizado e sem as aspas. No caso de documentos datilografados, deve-se
observar apenas o recuo \cite[5.3]{NBR10520:2002}.
\end{citacao}

\index{citações!simples}Citações simples, com até três linhas, devem ser
incluídas com aspas. Observe que em \LaTeX~ as aspas iniciais são diferentes das finais: ``Amor é fogo que
arde sem se ver''. 


% ---
\section{Remissões internas}
% ---

Ao nomear a \autoref{tab-nivinv}, apresentamos um exemplo de remissão interna,
que também pode ser feita quando indicamos o \autoref{cap_exemplos}\footnote{O
número do capítulo indicado é
\ref{cap_exemplos}, que se inicia à página \pageref{cap_exemplos}.}
(\nameref{cap_exemplos}, \autopageref{cap_exemplos}), por exemplo.

% ---
\section{Tabelas}
% ---

Apresenta-se um exemplo de tabela a ser confeccionada. Atente-se para as normas de tabela exigidas pela Universidade.

\index{tabelas}A \autoref{tab-nivinv} é um exemplo de tabela construída em
\LaTeX.

\begin{table}[htb]
\footnotesize
\caption[Níveis de investigação]{Níveis de investigação.}
\label{tab-nivinv}
\begin{tabular}{p{2.6cm}|p{6.0cm}|p{2.25cm}|p{3.40cm}}
  %\hline
   \textbf{Nível de Investigação} & \textbf{Insumos}  & \textbf{Sistemas de Investigação}  & \textbf{Produtos}  \\
    \hline
    Meta-nível & Filosofia\index{filosofia} da Ciência  & Epistemologia &
    Paradigma  \\
    \hline
    Nível do objeto & Paradigmas do metanível e evidências do nível inferior &
    Ciência  & Teorias e modelos \\
    \hline
    Nível inferior & Modelos e métodos do nível do objeto e problemas do nível inferior & Prática & Solução de problemas  \\
   % \hline
\end{tabular}
\legend{Fonte: \citeonline{van86}}
\end{table}

% ---
\section{Expressões matemáticas}
% ---

\index{expressões matemáticas}Use o ambiente \texttt{equation} para escrever
expressões matemáticas numeradas:

\begin{equation}
  \forall x \in X, \quad \exists \: y \leq \epsilon
\end{equation}

Escreva expressões matemáticas entre \$ e \$, como em $ \lim_{x \to \infty}
\exp(-x) = 0 $, para que fiquem na mesma linha.

Também é possível usar colchetes para indicar o início de uma expressão
matemática que não é numerada.

\[
\left|\sum_{i=1}^n a_ib_i\right|
\le
\left(\sum_{i=1}^n a_i^2\right)^{1/2}
\left(\sum_{i=1}^n b_i^2\right)^{1/2}
\]

Consulte mais informações sobre expressões matemáticas em
\url{http://code.google.com/p/abntex2/w/edit/Referencias}.

\section{Figuras}

\index{figuras}Figuras podem ser criadas diretamente em \LaTeX,
como o exemplo da \autoref{fig_circulo}.

\begin{figure}[htb]
	\begin{center}
	\caption{\label{fig_circulo}A delimitação do espaço.}
	    \setlength{\unitlength}{5cm}
		\begin{picture}(1,1)
		\put(0,0){\line(0,1){1}}
		\put(0,0){\line(1,0){1}}
		\put(0,0){\line(1,1){1}}
		\put(0,0){\line(1,2){.5}}
		\put(0,0){\line(1,3){.3333}}
		\put(0,0){\line(1,4){.25}}
		\put(0,0){\line(1,5){.2}}
		\put(0,0){\line(1,6){.1667}}
		\put(0,0){\line(2,1){1}}
		\put(0,0){\line(2,3){.6667}}
		\put(0,0){\line(2,5){.4}}
		\put(0,0){\line(3,1){1}}
		\put(0,0){\line(3,2){1}}
		\put(0,0){\line(3,4){.75}}
		\put(0,0){\line(3,5){.6}}
		\put(0,0){\line(4,1){1}}
		\put(0,0){\line(4,3){1}}
		\put(0,0){\line(4,5){.8}}
		\put(0,0){\line(5,1){1}}
		\put(0,0){\line(5,2){1}}
		\put(0,0){\line(5,3){1}}
		\put(0,0){\line(5,4){1}}
		\put(0,0){\line(5,6){.8333}}
		\put(0,0){\line(6,1){1}}
		\put(0,0){\line(6,5){1}}
		\end{picture}
	\end{center}
	\legend{Fonte: os autores}
\end{figure}

Ou então figuras podem ser incorporadas de arquivos externos, como é o caso da
\autoref{fig_grafico}. Se a figura que ser incluída se tratar de um diagrama, um
gráfico ou uma ilustração que você mesmo produza, priorize o uso de imagens
vetoriais no formato PDF. Com isso, o tamanho do arquivo final do trabalho será
menor, e as imagens terão uma apresentação melhor, principalmente quando
impressas, uma vez que imagens vetorias são perfeitamente escaláveis para
qualquer dimensão. Nesse caso, se for utilizar o Microsoft Excel para produzir
gráficos, ou o Microsoft Word para produzir ilustrações, exporte-os como PDF e
os incorpore ao documento conforme o exemplo abaixo. No entanto, para manter a
coerência no uso de software livre (já que você está usando \LaTeX e \abnTeX),
teste a ferramenta \textsf{InkScape}\index{InkScape}
(\url{http://inkscape.org/}). Ela é uma excelente opção de código-livre para
produzir ilustrações vetoriais, similar ao CorelDraw\index{CorelDraw} ou ao Adobe
Illustrator\index{Adobe Illustrator}. De todo modo, caso não seja possível
utilizar arquivos de imagens como PDF, utilize qualquer outro formato, como
JPEG, GIF, BMP, etc. Nesse caso, você pode tentar aprimorar as imagens
incorporadas com o software livre \textsf{Gimp}\index{Gimp}
(\url{http://www.gimp.org/}). Ele é uma alternativa livre ao Adobe
Photoshop\index{Adobe Photoshop}.

\begin{figure}[htb]
\caption{\label{fig_grafico}Gráfico produzido em Excel e salvo como PDF.}
	\begin{center}
	    \includegraphics[scale=0.5]{abntex2-modelo-img-grafico.pdf}
	\end{center}
	\legend{Fonte: \citeonline[p. 24]{araujo2012}}
\end{figure}

% ---
\section{Enumerações: alíneas e subalíneas}
% ---

\index{alíneas}\index{subalíneas}\index{incisos}Quando for necessário enumerar
os diversos assuntos de uma seção que não possua título, esta deve ser
subdividida em alíneas \cite[4.2]{NBR6024:2012}:

\begin{alineas}

  \item os diversos assuntos que não possuam título próprio, dentro de uma mesma
  seção, devem ser subdivididos em alíneas\footnote{As notas devem ser digitadas ou datilografadas
  dentro das margens, ficando separadas do texto por um espaço simples de entre as
  linhas e por filete de 5 cm, a partir da margem esquerda. Devem ser
  alinhadas, a partir da segunda linha da mesma nota, abaixo da primeira letra
  da primeira palavra, de forma a destacar o expoente, sem espaço entre elas e
  com fonte menor. \citeonline[5.2.1]{NBR14724:2011}}; 
  
  \item o texto que antecede as alíneas termina em dois pontos;
  \item as alíneas devem ser indicadas alfabeticamente, em letra minúscula,
  seguida de parêntese. Utilizam-se letras dobradas, quando esgotadas as
  letras do alfabeto;

  \item as letras indicativas das alíneas devem apresentar recuo em relação à
  margem esquerda;

  \item o texto da alínea deve começar por letra minúscula e terminar em
  ponto-e-vírgula, exceto a última alínea que termina em ponto final;

  \item o texto da alínea deve terminar em dois pontos, se houver subalínea;

  \item a segunda e as seguintes linhas do texto da alínea começa sob a
  primeira letra do texto da própria alínea;
  
  \item subalíneas \cite[4.3]{NBR6024:2012} devem ser conforme as alíneas a
  seguir:

  \begin{alineas}
     \item as subalíneas devem começar por travessão seguido de espaço;

     \item as subalíneas devem apresentar recuo em relação à alínea;

     \item o texto da subalínea deve começar por letra minúscula e terminar em
     ponto-e-vírgula. A última subalínea deve terminar em ponto final, se não
     houver alínea subsequente;

     \item a segunda e as seguintes linhas do texto da subalínea começam sob a
     primeira letra do texto da própria subalínea.
  \end{alineas}
  
  \item no \abnTeX\ estão disponíveis os ambientes \texttt{incisos} e
  \texttt{subalineas}, que em suma são o mesmo que se criar outro nível de
  \texttt{alineas}, como nos exemplos à seguir:
  
  \begin{incisos}
    \item \textit{Um novo inciso em itálico};
  \end{incisos}
  
  \item Alínea em \textbf{negrito}:
  
  \begin{subalineas}
    \item \textit{Uma subalínea em itálico};
    \item \underline{\textit{Uma subalínea em itálico e sublinhado}}; 
  \end{subalineas}
  
  \item Última alínea com \emph{ênfase}.
  
\end{alineas}


% ---
\section{Inclução de outros arquivos}\label{sec-include}
% ---

É uma boa prática dividir o seu documento em diversos arquivos, e não
apenas escrever tudo em um único. Esse recurso foi utilizado neste
documento. Para incluir diferentes arquivos em um arquivo principal,
de modo que cada arquivo incluído fique em uma página diferente, utilize o
comando:

\begin{verbatim}
   \include{documento-a-ser-incluido}      % sem a extensão .tex
\end{verbatim}

Para incluir documentos sem quebra de páginas, utilize:

\begin{verbatim}
   \input{documento-a-ser-incluido}      % sem a extensão .tex
\end{verbatim}

% ---
\section{Compilar o documento \LaTeX}
% ---

Geralmente os editores \LaTeX, como o
TeXlipse\footnote{\url{http://texlipse.sourceforge.net/}}, o
Texmaker\footnote{\url{http://www.xm1math.net/texmaker/}}, entre outros,
compilam os documentos automaticamente, de modo que você não precisa se
preocupar com isso.

No entanto, você pode compilar os documentos \LaTeX usando os seguintes
comandos, que devem ser digitados no \emph{Prompt de Comandos} do Windows ou no
\emph{Terminal} do Mac ou do Linux:

\begin{verbatim}
   pdflatex ARQUIVO_PRINCIPAL.tex
   bibtex ARQUIVO_PRINCIPAL.aux
   makeindex ARQUIVO_PRINCIPAL.idx 
   makeindex ARQUIVO_PRINCIPAL.nlo -s nomencl.ist -o ARQUIVO_PRINCIPAL.nls
   pdflatex ARQUIVO_PRINCIPAL.tex
   pdflatex ARQUIVO_PRINCIPAL.tex
\end{verbatim}

% ---
\section{Divisões do documento: seção}\label{sec-divisoes}
% ---

Esta seção testa o uso de divisões de documentos. Isto é uma seção.

\subsection{Divisões do documento: subseção}

Isto é uma subseção.

\subsubsection{Divisões do documento: subsubseção}

Isto é uma subsubseção.

\subsubsection{Divisões do documento: subsubseção}

Isto é outra subsubseção.

\subsection{Divisões do documento: subseção}\label{sec-exemplo-subsec}

Isto é uma subseção.

\subsubsection{Divisões do documento: subsubseção}

Isto é mais uma subsubseção da \autoref{sec-exemplo-subsec}.

% ---
\section{Este é um exemplo de nome de seção longo. Ele deve estar
alinhado à esquerda e a segunda e demais linhas devem iniciar logo abaixo da
primeira palavra da primeira linha}
% ---

Isso atende à norma \citeonline[seções de 5.2.2 a 5.2.4]{NBR14724:2011} 
 e \citeonline[seções de 3.1 a 3.8]{NBR6024:2012}.


% ---
\section{Consulte o manual da classe \textsf{abntex2}}
% ---

Consulte o manual da classe \textsf{abntex2} \cite{abntex2classe} para uma
referência completa das macros e ambientes disponíveis. Além disso, o manual
possui informações adicionais sobre as normas ABNT observadas pelo \abnTeX.

%% ------------------------------------------------------------------------
%% Conclusão
\chapter[Conclusão]{Conclusão}
A conclusão também é um capítulo.
%% ========================================================================

%% ========================================================================
%% ELEMENTOS PÓS-TEXTUAIS
%% ------------------------------------------------------------------------
\postextual
%% ========================================================================

%% ========================================================================
%% Referências bibliográficas
%% ------------------------------------------------------------------------
\bibliography{abntex2-modelo-references}
%% ========================================================================

%% ========================================================================
%% Glossário
%% ------------------------------------------------------------------------
%\glossary
%% ========================================================================

%% ========================================================================
%% Apêndices
%% ------------------------------------------------------------------------
%% Inicia os apêndices
%% ------------------------------------------------------------------------
\begin{apendicesenv}
\partapendices %% Imprime uma página indicando o início dos apêndices
\chapter{Quisque libero justo} %% Divisão em capítulos, como no restante
\lipsum[1-5]
\end{apendicesenv}
%% ========================================================================

%% ========================================================================
%% Anexos
%% ------------------------------------------------------------------------
%% Inicia os anexos
%% ------------------------------------------------------------------------
\begin{anexosenv}
\partanexos  %% Imprime uma página indicando o início dos anexos
\chapter{Morbi ultrices rutrum lorem.} %% Divisão em capítulos.
\lipsum[1-25]
\section{Test} %% Divisão em sessões.
\lipsum[1-20]
\end{anexosenv}
%% ========================================================================

\end{document}